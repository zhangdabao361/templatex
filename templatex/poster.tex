%%
%% This is file `tikzposter-template.tex',
%% generated with the docstrip utility.
%%
%% The original source files were:
%%
%% tikzposter.dtx  (with options: `tikzposter-template.tex')
%%
%% This is a generated file.
%%
%% Copyright (C) 2014 by Pascal Richter, Elena Botoeva, Richard Barnard, and Dirk Surmann
%%
%% This file may be distributed and/or modified under the
%% conditions of the LaTeX Project Public License, either
%% version 2.0 of this license or (at your option) any later
%% version. The latest version of this license is in:
%%
%% http://www.latex-project.org/lppl.txt
%%
%% and version 2.0 or later is part of all distributions of
%% LaTeX version 2013/12/01 or later.
%%


\documentclass{tikzposter} %Options for format can be included here

\usepackage{todonotes}

\usepackage[tikz]{bclogo}
\usepackage{lipsum}
\usepackage{amsmath}

\usepackage{booktabs}
\usepackage{longtable}
\usepackage[absolute]{textpos}
\usepackage[it]{subfigure}
\usepackage{graphicx}
\usepackage{cmbright}
%\usepackage[default]{cantarell}
%\usepackage{avant}
%\usepackage[math]{iwona}
\usepackage[math]{kurier}
\usepackage[T1]{fontenc}


%% add your packages here
\usepackage{hyperref}
% for random text
\usepackage{lipsum}
\usepackage[english]{babel}
\usepackage[pangram]{blindtext}

\colorlet{backgroundcolor}{blue!10}

 % Title, Author, Institute
\title{flip 01 proiect report}
\author{Baojie Zhang}
\institute{ Xi'an Shiyou University, China}
%\titlegraphic{logos/tulip-logo.eps}

%Choose Layout
\usetheme{Wave}

%\definebackgroundstyle{samplebackgroundstyle}{
%\draw[inner sep=0pt, line width=0pt, color=red, fill=backgroundcolor!30!black]
%(bottomleft) rectangle (topright);
%}
%
%\colorlet{backgroundcolor}{blue!10}

\begin{document}


\colorlet{blocktitlebgcolor}{blue!23}

 % Title block with title, author, logo, etc.
\maketitle

\begin{columns}
 % FIRST column
\column{0.5}% Width set relative to text width

%%%%%%%%%% -------------------------------------------------------------------- %%%%%%%%%%
 %\block{Main Objectives}{
%  	      	\begin{enumerate}
%  	      	\item Formalise research problem by extending \emph{outlying aspects mining}
%  	      	\item Proposed \emph{GOAM} algorithm is to solve research problem
%  	      	\item Utilise pruning strategies to reduce time complexity
%  	      	\end{enumerate}
%%  	      \end{minipage}
%}
%%%%%%%%%% -------------------------------------------------------------------- %%%%%%%%%%


%%%%%%%%%% -------------------------------------------------------------------- %%%%%%%%%%
\block{Introduction}{
    Picture yourself strolling through your local, open-air market... What do you see? What do you smell? What will you make for dinner tonight?
    If you're in Northern California, you'll be walking past the inevitable bushels of leafy greens, spiked with dark purple kale and the bright pinks and yellows of chard. Across the world in South Korea, mounds of bright red kimchi greet you, while the smell of the sea draws your attention to squids squirming nearby. 
    India’s market is perhaps the most colorful, awash in the rich hues and aromas of dozens of spices: turmeric, star anise, poppy seeds, and garam masala as far as the eye can see.
    The subject requirements are asks to predict the category of a dish's cuisine given a list of its ingredients.
  	In the dataset, we include the recipe id, the type of cuisine, and the list of ingredients of each recipe (of variable length). The data is stored in JSON format. \\
  	% \begin{description}
  	% \item[Outlying Aspects Mining] aims to identify a subspace
    % which makes the query object most outlying,
    % rather than verifying whether it is an outlier or not.
    % The task of \emph{Outlying Aspects Mining}
    % is to explain which aspects make the query object most different.
    \vspace{.5cm}
  	\begin{tabular}{ c | c | c }
      \toprule
      Name & Description & Attribute      \\
      \midrule
      train.json & training set(the type of cuisine, and & Data: id, cuisine,\\     
                 & the list of ingredients of each recipe) & ingredients\\
     \midrule
       test.json & Test set(predict the cuisine type & Data:id,ingredients     \\
                 & of the list ingredients) &     \\
      % sample_submission.csv & a sample submission file & Data:id,cuisine    \\
      \midrule
      sample.csv & a sample submission file & Data:id,ingredients     \\
       \bottomrule
  \end{tabular}
  	% \item[Outlier Detection] aims to identify all possible outliers in the dataset,
    % without explaining why or how they are different.
    % Hence,
    % the outlying aspects mining is also referred to
    % \emph{outlier interpretation}
    % or \emph{object explanation}.
    % \end{description}
    

  	% In this paper,
    % we extend the task of \emph{outlying aspects mining} to the \emph{group} level,
    % formalize the research problem of \emph{group outlying aspects mining},
    % and propose a novel algorithm named GOAM to solve the
    % \emph{group outlying aspects mining} problem.
}
%%%%%%%%%% -------------------------------------------------------------------- %%%%%%%%%%


%%%%%%%%%% -------------------------------------------------------------------- %%%%%%%%%%
\block{Data Analysis}{
\begin{itemize}
    \item
    There are 39774 data in the training set.\\
    There are 9944 data in the test set.\\   

    \item
    Data is imported as DataFrame object, each recipe is a separate line.\\  
    \item  There are no missing values in the training set.\\
    \item The following results are obtained by sorting and sorting the data of ingredients:

\end{itemize}
\begin{center}
  \vspace{.5cm}
  	\begin{tabular}{ c | c }
      \toprule
      id & cuisine      \\
      \midrule
      salt & 18049  \\
    %  \midrule
     onions	& 797 \\
      % \midrule
      olive oil	& 7972 \\
      water	& 7457\\
      garlic	& 7380\\
       \bottomrule
  \end{tabular}
\end{center}
% \begin{itemize}
%   \item We can also get the dish label from the training set,as follows:\\
%         'greek', 'filipino', 'indian', 'jamaican', 'spanish', 'italian', 'mexican', 'chinese', 'british', 'thai', 'vietnamese', 'cajun_creole', 'brazilian',
%   %  'french', 'japanese', 'irish', 'korean', 'moroccan', 'russian'.\\ 
% \end{itemize}   

% \begin{center}
%     \begin{minipage}{0.3\linewidth}
%     \centering
%     \begin{tikzfigure}
%     \missingfigure[figcolor=white]{Testing figcolor}
%     {\small{Group Outlying Aspects Mining}}
%     \end{tikzfigure}%
%     \end{minipage}
%     \hfill
%     \begin{minipage}{0.3\linewidth}
%     \centering
%     \begin{tikzfigure}
%     \missingfigure[figcolor=white]{Testing figcolor}
%     {\small{Outlying Aspects Mining}}
%     \end{tikzfigure}%
%     \end{minipage}
%     \hfill
%     \begin{minipage}{0.3\linewidth}
%     \centering
%     \begin{tikzfigure}
%     \missingfigure[figcolor=white]{Testing figcolor}
%     {\small{Outlier Detection}}
%     \end{tikzfigure}%
%     \end{minipage}
% \end{center}
}
%%%%%%%%%% -------------------------------------------------------------------- %%%%%%%%%%


%%%%%%%%%% -------------------------------------------------------------------- %%%%%%%%%%

%\note{Note with default behavior}

%\note[targetoffsetx=12cm, targetoffsety=-1cm, angle=20, rotate=25]
%{Note \\ offset and rotated}

 % First column - second block


%%%%%%%%%% -------------------------------------------------------------------- %%%%%%%%%%
\block{Data Description}{
  We can find the percentage of each country’s dishes in the total test set.
 % \begin{table} 
    %\begin{center}
      % \includegraphics[width=1.0\linewidth,height=.4\linewidth]{C:/Users/lenovo/Desktop/01/slides/Figure/Figure_1.eps}\\
      %   {image1:The percentage of dishes}\\
  % \end{table}
%    1) Group Feature Extraction,
%    2) Outlying Degree Scoring, and
%    3) Outlying Aspects Identification.
  	
% \begin{tikzfigure}%[Overall architecture of \emph{GOAM} algorithm]
% %  \includegraphics[width=0.8\linewidth]{figures//framework.pdf}
%     \missingfigure[figcolor=white]{Testing figcolor}
% \end{tikzfigure}
		
\begin{description}
  	\item
    We can find the percentage of each country’s dishes in the total test set. 
    At this point, we will find that in the test set, Italian dishes account for 
    the largest percentage, more tahn Nineteen percent in the training set, 
    followed by Mexican dishes, which more than sixteen percent in the training set, 
    and the following order is: \\
    Itian, Indian,Chinese, French,Cajun creole, thai, 
    japanese, greek, spanish, korean, vietnamese, moroccan,british,fillplno, ifish, jamaican,
    russian, brazilian.
%    \item
%    The histogram of $G_q$ on three features are as follows.
\end{description}

% \begin{center}
%     \begin{minipage}{0.3\linewidth}
%     \centering
%     \begin{tikzfigure}
%     \missingfigure[figcolor=white]{Testing figcolor}
%     {\small{Histogram of $G_q$ on $f_1$}}
%     \end{tikzfigure}%
%     \end{minipage}
%     \hfill
%     \begin{minipage}{0.3\linewidth}
%     \centering
%     \begin{tikzfigure}
%     \missingfigure[figcolor=white]{Testing figcolor}
%     {\small{Histogram of $G_q$ on $f_2$}}
%     \end{tikzfigure}%
%     \end{minipage}
%     \hfill
%     \begin{minipage}{0.3\linewidth}
%     \centering
%     \begin{tikzfigure}
%     \missingfigure[figcolor=white]{Testing figcolor}
%     {\small{Histogram of $G_q$ on $f_3$}}
%     \end{tikzfigure}%
%     \end{minipage}
% \end{center}
% \begin{description}
% \item[Outlying Degree Scoring]
%     In this step,
%     we first calculate the \emph{earth mover distance} (EMD) of one feature among different groups.
%     The earth mover distance reflects the minimum mean distance
%     between groups on one feature.
%     So,
%     we utilize the EMD to measure the difference between groups of each feature.
% \end{description}
}
%%%%%%%%%% -------------------------------------------------------------------- %%%%%%%%%%


% SECOND column
\column{0.5}
 %Second column with first block's top edge aligned with with previous column's top.

%%%%%%%%%% -------------------------------------------------------------------- %%%%%%%%%%
\block{}{
\begin{description}
    \item
    We can find the ingredients that account for the most ingredients\\ 
    in each country’s dishes:

\end{description}

% \begin{table} 
  %\begin{center}
    %\includegraphics[width=1.0\linewidth,height=.4\linewidth]{C:/Users/lenovo/Desktop/01/slides/Figure/Figure_1.eps}\\
     % {image1:The percentage of dishes}\\
     \begin{minipage}{1\linewidth}
      \centering
      \includegraphics[width=1\textwidth]{C:/Users/lenovo/Desktop/01/slides/Figure/Figure_3.JPG} 
  \end{minipage}

% \end{table}
\begin{description}
    \item
    In the same way, we can find the 25 
    cooking ingredients or cooking essences with the least amount of each dish.

\end{description}
}
%%%%%%%%%% -------------------------------------------------------------------- %%%%%%%%%%
% Second column - first block


%%%%%%%%%% -------------------------------------------------------------------- %%%%%%%%%%
\block[titleleft]{Data Processing}
{
\begin{description}
    \item
    In the data processing stage, we first need to process the text data. 
    Since the initial data type we get is dataFrame type data, we first 
    need to process the data and convert the original list type data to string type.
\end{description}
% \vspace{.5cm}
% \begin{tabular}{ c | c | c | c }
%     \toprule
%     Method     &  Truth Outlying Aspects    & Identified Aspects & Accuracy      \\
%     \midrule
%     GOAM       &  $\{F_1\}$, $\{F_2F_4\}$   &  $\{F_1\}$, $\{F_2F_4\}$    & 100\%    \\

%      Arithmetic Mean based OAM &  $\{F_1\}$, $\{F_2F_4\}$   &  $\{F_4\}$, $\{F_2\}$    &  0\% \\

%      Median based OAM &  $\{F_1\}$, $\{F_2F_4\}$   &  $\{F_2\}$, $\{F_4\}$    &           0\% \\
%      \bottomrule
% \end{tabular}
% \vspace{.2cm}
\begin{description}
    \item
    After the data is converted, we need to use TF-IDF to calculate the degree of 
    similarity between each ingredient and the dish, on this basis, we can find 
    dishes with more similarities:
\end{description}
\begin{minipage}{1\linewidth}
  \centering
  \includegraphics[width=0.6\textwidth]{C:/Users/lenovo/Desktop/01/slides/Figure/Figure_2.JPG} 
\end{minipage}

\begin{description}
  \item
  Next, we can use Linear SVC to predict the dishes:
  For the first modeling synthesis, its best params are'C': 1,'loss':'hinge','penalty':'l2'\\
  The best result we can get at this time is 0.7857646647354912.

\end{description}

% \begin{description}
% \item[NBA Dataset] was collected from Yahoo Sports
% website (\url{http://sports.yahoo.com.cn/nba}).
% The data include all teams from the six divisions,
% and each player in the team has $12$ features.
% \end{description}
% \vspace{.5cm}
% \begin{tabular}{ c | c | c }
%     \toprule
%     Teams                   & Trivial Outlying Aspects  & NonTrivial Outlying Aspects    \\
%     \toprule
%     Cleveland Cavaliers     & \{3FA\}                   & \{FGA, FT\%\}, \{FGA, FG\%\} \\
%     Orlando Magic           & \{Stl\}                   & None                         \\
%     Milwaukee Bucks         & \{To\}, \{FTA\}           & \{FGA, FTA\}, \{3FA, FTA\}     \\
% %    Golden State Warriors   & \{FG\%\}                  & \{FT\%, Blk\}, \{FGA, 3PT\%, FTA\}\\
% %    Utah Jazz               & \{Blk\}                   & \{3FA, 3PT\%\}                    \\
%     New Orleans Pelicans    & \{FT\%\}, \{FTA\}         & \{FTA, Stl\}, \{FTA, To\}          \\
%     \bottomrule
% \end{tabular}
           
% \begin{minipage}{0.5\linewidth}
%     \centering
%     \begin{tikzfigure}
%     \missingfigure[figcolor=white]{Testing figcolor}

%     {\small{New Orleans Pelicans on FT\%}}
%     \end{tikzfigure}%
% \end{minipage}
% \hfill
% \begin{minipage}{0.5\linewidth}
%     \centering
%     \begin{tikzfigure}
%     \missingfigure[figcolor=white]{Testing figcolor}

%     {\small{New Orleans Pelicans on FTA}}
%     \end{tikzfigure}%
% \end{minipage}
% \vspace{.2cm}
% \begin{description}
% \item
% \texttt{New Orleans Pelicans} has more players with
% lower \{free throw percentage\}, \{free throws attempted\}.
% \end{description}
}
%%%%%%%%%% -------------------------------------------------------------------- %%%%%%%%%%


% Second column - second block
%%%%%%%%%% -------------------------------------------------------------------- %%%%%%%%%%
\block[titlewidthscale=1, bodywidthscale=1]
{Conclusion}
{
\begin{description}
  \item 
  In this practice, I learned how to process text-related materials. 
  When processing text, it is roughly divided into the following steps: 
  1. Text preprocessing; 2. Text representation; 3. Spatial dimensionality reduction; 
  4. Classification model training; 5. Classification performance evaluation. \\
  In the above steps, I still have deficiencies in the two aspects of spatial 
  dimensionality reduction and classification performance evaluation. 
  In the subsequent learning, I will pay attention to strengthening the learning 
  of these two aspects.In addition, in the process of processing the data, I also 
  tried to use the bag of words model, and here is a word cloud, which is a very interesting attempt.\\
  
  In general, I have learned a lot during this practice, but there are also many shortcomings, 
  which need to be filled in the next time.
  % \item[Strategies]
  % Utilize the pruning strategies to \\ reduce time complexity.
\end{description}
}
%%%%%%%%%% -------------------------------------------------------------------- %%%%%%%%%%


% Bottomblock
%%%%%%%%%% -------------------------------------------------------------------- %%%%%%%%%%
\colorlet{notebgcolor}{blue!20}
\colorlet{notefrcolor}{blue!20}
% \note[targetoffsetx=8cm, targetoffsety=-4cm, angle=30, rotate=15,
% radius=2cm, width=.26\textwidth]{
% Acknowledgement
% \begin{itemize}
%     \item
%     International Cooperation Project (Y7Z0511101)
%     of IIE,
%     Chinese Academy of Sciences
%  \end{itemize}
% }

%\note[targetoffsetx=8cm, targetoffsety=-10cm,rotate=0,angle=180,radius=8cm,width=.46\textwidth,innersep=.1cm]{
%Acknowledgement
%}

%\block[titlewidthscale=0.9, bodywidthscale=0.9]
%{Acknowledgement}{
%}
%%%%%%%%%% -------------------------------------------------------------------- %%%%%%%%%%

\end{columns}


%%%%%%%%%% -------------------------------------------------------------------- %%%%%%%%%%
%[titleleft, titleoffsetx=2em, titleoffsety=1em, bodyoffsetx=2em,%
%roundedcorners=10, linewidth=0mm, titlewidthscale=0.7,%
%bodywidthscale=0.9, titlecenter]

%\colorlet{noteframecolor}{blue!20}
\colorlet{notebgcolor}{blue!20}
\colorlet{notefrcolor}{blue!20}
\note[targetoffsetx=-13cm, targetoffsety=-12cm,rotate=0,angle=180,radius=8cm,width=.96\textwidth,innersep=.4cm]
{
\begin{minipage}{0.3\linewidth}
\centering
\includegraphics[width=24cm]{logos/tulip-wordmark.eps}
\end{minipage}
\begin{minipage}{0.7\linewidth}
{ \centering
 The $11^{th}$ International Conference on Knowledge Science,
  Engineering and Management (KSEM 2018),
  17-19/08/2018, Changchun, China
}
\end{minipage}
}
%%%%%%%%%% -------------------------------------------------------------------- %%%%%%%%%%


\end{document}

%\endinput
%%
%% End of file `tikzposter-template.tex'.
