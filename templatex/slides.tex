%
% ---------------------------------------------------------------
% Copyright (C) 2012-2018 Gang Li
% ---------------------------------------------------------------
%
% This work is the default powerdot-tuliplab style test file and may be
% distributed and/or modified under the conditions of the LaTeX Project Public
% License, either version 1.3 of this license or (at your option) any later
% version. The latest version of this license is in
% http://www.latex-project.org/lppl.txt and version 1.3 or later is part of all
% distributions of LaTeX version 2003/12/01 or later.
%
% This work has the LPPL maintenance status "maintained".
%
% This Current Maintainer of this work is Gang Li.
%
%

\documentclass[
 size=14pt,
 paper=smartboard,  %a4paper, smartboard, screen
 mode=present, 		%present, handout, print
 display=slides, 	% slidesnotes, notes, slides
 style=tuliplab,  	% TULIP Lab style
 pauseslide,
 fleqn,leqno]{powerdot}


\usepackage{cancel}
\usepackage{caption}
\usepackage{stackengine}
\usepackage{smartdiagram}
\usepackage{attrib}
\usepackage{amssymb}
\usepackage{amsmath} 
\usepackage{amsthm} 
\usepackage{mathtools}
\usepackage{rotating}
\usepackage{graphicx}
\usepackage{boxedminipage}
\usepackage{rotate}
\usepackage{calc}
\usepackage[absolute]{textpos}
\usepackage{psfrag,overpic}
\usepackage{fouriernc}
\usepackage{pstricks,pst-3d,pst-grad,pstricks-add,pst-text,pst-node,pst-tree}
\usepackage{moreverb,epsfig,subfigure}
\usepackage{color}
\usepackage{booktabs}
\usepackage{etex}
\usepackage{breqn}
\usepackage{multirow}
\usepackage{natbib}
\usepackage{bibentry}
\usepackage{gitinfo2}
\usepackage{siunitx}
\usepackage{nicefrac}
%\usepackage{geometry}
%\geometry{verbose,letterpaper}
\usepackage{media9}
\usepackage{animate}
%\usepackage{movie15}
\usepackage{auto-pst-pdf}

\usepackage{breakurl}
\usepackage{fontawesome}
\usepackage{xcolor}
\usepackage{multicol}



\usepackage{verbatim}
\usepackage[utf8]{inputenc}
\usepackage{dtk-logos}
\usepackage{tikz}
\usepackage{adigraph}
%\usepackage{tkz-graph}
\usepackage{hyperref}
%\usepackage{ulem}
\usepackage{pgfplots}
\usepackage{verbatim}
\usepackage{fontawesome}


\usepackage{todonotes}
% \usepackage{pst-rel-points}
\usepackage{animate}
\usepackage{fontawesome}

\usepackage{listings}
\lstset{frameround=fttt,
frame=trBL,
stringstyle=\ttfamily,
backgroundcolor=\color{yellow!20},
basicstyle=\footnotesize\ttfamily}
\lstnewenvironment{code}{
\lstset{frame=single,escapeinside=`',
backgroundcolor=\color{yellow!20},
basicstyle=\footnotesize\ttfamily}
}{}


\usepackage{hyperref}
\hypersetup{ % TODO: PDF meta Data
  pdftitle={Presentation Title},
  pdfauthor={Gang Li},
  pdfpagemode={FullScreen},
  pdfborder={0 0 0}
}


% \usepackage{auto-pst-pdf}
% package to show source code

\definecolor{LightGray}{rgb}{0.9,0.9,0.9}
\newlength{\pixel}\setlength\pixel{0.000714285714\slidewidth}
\setlength{\TPHorizModule}{\slidewidth}
\setlength{\TPVertModule}{\slideheight}
\newcommand\highlight[1]{\fbox{#1}}
\newcommand\icite[1]{{\footnotesize [#1]}}

\newcommand\twotonebox[2]{\fcolorbox{pdcolor2}{pdcolor2}
{#1\vphantom{#2}}\fcolorbox{pdcolor2}{white}{#2\vphantom{#1}}}
\newcommand\twotoneboxo[2]{\fcolorbox{pdcolor2}{pdcolor2}
{#1}\fcolorbox{pdcolor2}{white}{#2}}
\newcommand\vpspace[1]{\vphantom{\vspace{#1}}}
\newcommand\hpspace[1]{\hphantom{\hspace{#1}}}
\newcommand\COMMENT[1]{}

\newcommand\placepos[3]{\hbox to\z@{\kern#1
        \raisebox{-#2}[\z@][\z@]{#3}\hss}\ignorespaces}

\renewcommand{\baselinestretch}{1.2}


\newcommand{\draftnote}[3]{
	\todo[author=#2,color=#1!30,size=\footnotesize]{\textsf{#3}}	}
% TODO: add yourself here:
%
\newcommand{\gangli}[1]{\draftnote{blue}{GLi:}{#1}}
\newcommand{\shaoni}[1]{\draftnote{green}{sn:}{#1}}
\newcommand{\gliMarker}
	{\todo[author=GLi,size=\tiny,inline,color=blue!40]
	{Gang Li has worked up to here.}}
\newcommand{\snMarker}
	{\todo[author=Sn,size=\tiny,inline,color=green!40]
	{Shaoni has worked up to here.}}

%%%%%%%%%%%%%%%%%%%%%%%%%%%%%%%%%%%%%%%%%%%%%%%%%%%%%%%%%%%%%%%%%%%%%%%%
% title
% TODO: Customize to your Own Title, Name, Address
%
\title{Summary and plan for next week}
\author{
Baojie  Zhang
\\
\\Xi'an Shiyou University
% \\Deakin University
% \\Chinese Academy of Sciences
}
\date{\gitCommitterDate}


% Customize the setting of slides
\pdsetup{
% TODO: Customize the left footer, and right footer
rf=\href{http://www.tulip.org.au}{
Last Changed by: \textsc{\gitCommitterName}\ \gitVtagn-\gitAbbrevHash\ (\gitAuthorDate)
},
cf={Summary and plan for next week},
}


\begin{document}

\maketitle

%\begin{slide}{Overview}
%\tableofcontents[content=sections]
%\end{slide}


%%==========================================================================================
%%
\begin{slide}[toc=,bm=]{Overview}
\tableofcontents[content=currentsection,type=1]
\end{slide}
%%
%%==========================================================================================


\section{Summary}


%%==========================================================================================
%%
\begin{slide}{Work summary}

  \begin{itemize}
    \item Summary of work last week.\\  
    \begin{itemize}
      \item Understand the two subject directions:\\
            1.HTM Data mining\\
            2.Abnormal detection\\
      \item Learning video lesson: Data Mining: Theory and Algorithms\\
    \end{itemize}
  \end{itemize}
% \begin{center}
% \twotonebox{\rotatebox{90}{summary}}{\parbox{.86\textwidth}
% {Summary of work last week\\
% 1.
% % \begin{itemize}
% % \item In the dataset, we include the recipe id, the type of cuisine, and the list of ingredients of each recipe (of variable length). The data is stored in JSON format. 
% % % \item NBA coaches would prefer to
% % % find out the strengths and weaknesses of the player (a query object).
% % \end{itemize}
% }}


% \begin{table}
%   % \setlength{\abovecaptionskip}{0pt}
%   % \setlength{\belowcaptionskip}{10pt}
%   \centering
%   \caption{Data description}  


% \begin{tabular}{ccc}
%   \hline
%   Name & Description & Attribute    \\ 
%   \hline
%   train.json & training set(the type of cuisine, and & Data: id, cuisine,\\
%              & the list of ingredients of each recipe) & ingredients     \\
%   test.json & Test set(predict the cuisine type & Data:id,ingredients     \\
%             & of the list ingredients) &     \\
%   sample_submission.csv & a sample submission file in the  format & Data:id,cuisine    \\
%   \hline

% \end{tabular}
% \end{table}

% \end{center}


% \begin{center}
%   \twotonebox{\rotatebox{90}{Problem}}{\parbox{.86\textwidth}
%   {
%   \begin{itemize}
%   \item train.json:the training set containing recipes id, type of cuisine, and list of ingredients.
%   \item test.json:the test set containing recipes id, and list of ingredients.
%   % \item NBA coaches would prefer to
%   % find out the strengths and weaknesses of the player (a query object).
%   \end{itemize}
%   }}
  
%   \end{center}
% \bigskip
% \begin{center}
% \begin{tabular}{c| c c c c }
% \toprule
% Player & \texttt{3PT\%}  & \texttt{FTA} & \texttt{FT\%} & \texttt{To} \\
% \midrule
% $P_1$
% &  {$65$} &  {$4$} &  {$33$} &  {$8$} \\
% $P_2$
% &  {$78$} &  {$1$}&  {$65$}&  {$5$} \\
% $P_3$
% &  {$58$} &  {$6$} &  {$46$} &  {$3$} \\
% $P_4$
% &  {$68$} &  {$1.2$}&  {$85$}&  {$6.2$} \\
% $P_5$
% &  {$58$} &  {$6.2$} &  {$36$} &  {$3.4$}\\
% \bottomrule
% \end{tabular}
% \end{center}
% \bigskip

%%==========================================================================================
% \begin{note}
% First, I will introduce the problem definition.
% In the real life,
% a teacher may be interested in the characteristics that
% make one student obvious different from others.
% Or,
% NBA sports coaches would prefer to
% know the advantages and disadvantages of one player.
% Here, the player can be regarded as a query object.

% For example, team A has five players,
% each player has four features.
% The NBA sports coaches may want to know the features of
% player $1$ that are different from others.

% The above example can be seen as outlying aspects mining.
% The main purpose of outlying aspects mining is to identify
% the outstanding features of the query object.
% \end{note}
%%==========================================================================================

\end{slide}
%%
%%==========================================================================================


%%==========================================================================================
%%
% \begin{slide}[toc=,bm=]{}
% \begin{center}
% \begin{tabular}{c| c c c c }
% \toprule
% %\centering
% Player & \texttt{3PT\%}  & \texttt{FTA} & \texttt{FT\%} & \texttt{To} \\
% \midrule
% $P_1$
% &  {$65$} &  {$4$} &  {$33$} &  {$8$} \\
% $P_2$
% &  {$78$} &  {$1$}&  {$65$}&  {$5$} \\
% $P_3$
% &  {$58$} &  {$6$} &  {$46$} &  {$3$} \\
% $P_4$
% &  {$68$} &  {$1.2$}&  {$85$}&  {$6.2$} \\
% $P_5$
% &  {$58$} &  {$6.2$} &  {$36$} &  {$3.4$}\\
% \bottomrule
% \end{tabular}
% \end{center}

% \bigskip

% \twocolumn[
% \savevalue{lfrheight}=4.6cm,
% \savevalue{lfrprop}={
% linestyle=solid,framearc=.2,linewidth=1pt},
% rfrheight=\usevalue{lfrheight},
% rfrprop=\usevalue{lfrprop}
% ]{
% Outlying Aspects Mining
% \begin{itemize}
% \item
% \smallskip
% Explain the distinctive \textcolor{orange}{aspects} of the query object.
% \smallskip
% \item
% \smallskip
% The query object may (or may not) be an outlier.
% \end{itemize}
% }{
% Outlier Detection
% \begin{itemize}
% \item
% \smallskip
% Find out \textcolor{orange}{all} unusual
% \textcolor{orange}{objects} in the whole dataset.
% \smallskip
% \item
% \smallskip
% \textcolor{orange}{No} explanation on how they are different.
% \end{itemize}
% }

% %%==========================================================================================
% % \begin{note}
% % Based on the above example,
% % I will compare the differences
% % between outlying aspects mining and outlier detection.

% % Outlying aspects mining aims to
% % explain the distinctive aspects of the query object.
% % The query object may or may not be an outlier.
% % In contrast,
% % Outlier detection aims to discover all possible
% % outlying objects in the dataset.
% % Without explaining how and why they are different.

% % Let's go back to the NBA example,
% % in that example,
% % the output of the outlying aspects mining may be
% % a combination of four features,
% % but the output of the outlier detection may be any of those five players.
% % \end{note}
% %%==========================================================================================

% \end{slide}
% %%
%%==========================================================================================


%%==========================================================================================
%%
% \begin{slide}{Group Outlying Aspects Mining}
% \twotonebox {\rotatebox{90}{Defn}}{\parbox{.88\textwidth}
% {
% {\textcolor{orange}{Group outlying aspects mining} aims to
% identify the outstanding features of the group of query object.
% \begin{itemize}
% \item
% Doctors desire to identify the merits \& demerits between
% \textcolor{orange}{a group of cancer patients} and normal people.
% \item
% NBA coaches are passionate about exploring the obvious advantages \&
% disadvantages of \textcolor{orange}{the team}.
% \end{itemize}
% }
% }}

% \vspace{1.5cm}

% \twocolumn{
% \begin{figure}
%   \centering
%   \selectcolormodel{rgb}
%   \missingfigure{Testing.}
%   %\includegraphics[width=0.6\textwidth]{figures//demical.eps}\\
%   \caption{Medical}\label{fig:demical}
% \end{figure}
% }{
% \begin{figure}
%   \centering
%   \selectcolormodel{rgb}
%   \missingfigure{Testing.}
%   %\includegraphics[width=0.6\textwidth]{figures//NBA_team.eps}\\
%   \caption{NBA-Team}\label{fig:timg}
% \end{figure}
% }

% %%==========================================================================================
% \begin{note}
% However,
% there is such a phenomenon in real life.
% Doctors desire to identify the characteristics between
% a group of cancer patients and normal people.
% NBA coaches are passionate about exploring the obvious strengths and
% weaknesses of the team compared with other teams.

% Based on such a phenomenon in the real life,
% we proposed the concept of group outlying aspects mining.
% \end{note}
% %%==========================================================================================

% \end{slide}
% %%
% %%==========================================================================================


% %%==========================================================================================
% %%
% \begin{slide}[toc=,bm=]{Problem Formalization}
% \twotonebox {\rotatebox{90}{Defn}}{\parbox{.88\textwidth}
% {
% {\textcolor{orange}{Group outlying aspects mining} aims to identify
% the \textcolor{orange}{top-k group outlying subspace $s \subseteq F$} in
% which the query group $G_q$ is \textcolor{orange} {distinctive with other groups}.
% \begin{itemize}
% \item
% $G = \{G_q, G_2, G_3,..., G_n\}$ $\Leftrightarrow$ a set of groups.
% \item
% $G_q$ $\Leftrightarrow$ the query group.
% \item
% Other groups $\Leftrightarrow$ comparison groups.
% \item
% Each object in the group has $d$ features $F = \{f_1, f_2, ..., f_d\}$.
% \end{itemize}
% }
% }}

% %%==========================================================================================
% \begin{note}
% Next,
% let me talk about the concept of group outlying aspects mining.

% For example,
% Dataset $G$ has $n$ groups.
% $G_q$ is the query group.
% and other groups are comparison groups.
% Each object in the group has d features $F = $ $f_1$, $f_2$, $f_3$ to $f_d$.
% The group outlying aspects mining is to identify the top-k group outlying subspaces,
% which are different from other groups.

% What does the top-k group outlying subspaces mean?
% Next, I will explain it.
% \end{note}

% %%==========================================================================================
% \end{slide}
% %%
% %%==========================================================================================


% %%==========================================================================================
% %%
% \begin{slide}[toc=,bm=]{Term Definition}
% \begin{itemize}
% \item
% Top-k group outlying subspaces

% \begin{itemize}
% \item
% $\rho_s(\cdot)$ $\Rightarrow$ outlying scoring function.

% \item
% $\rho_s(\cdot)$ quantifies the outlying degree of the
% query group $G_q$ in the subspace $s$.

% \item
% Order by DESC using scoring function $\rho(\cdot)$
% to identify top K group outlying subspaces.
% \end{itemize}
% \end{itemize}

% \begin{figure}[htbp]
%     \centering
%     \subfigure[Original Feature Spaces]{
%       \selectcolormodel{rgb}
%       \missingfigure[figwidth=5.5cm]{Test.}
%         %\includegraphics[width=0.3\textwidth]{figures//example-basketball-original.eps}
%         \label{fig:basketball-original}
%     }
%     \subfigure[Group Outlying Spaces]{
%        \selectcolormodel{rgb}
%        \missingfigure[figwidth=5.5cm]{Test.}
%         %\includegraphics[width=0.3\textwidth]{figures//example-basketball-projection.eps}
%         \label{fig:basketball-projection}
%     }
%     \subfigure[Another Subspaces]{
%       \selectcolormodel{rgb}
%       \missingfigure[figwidth=5.5cm]{Test.}
%         %\includegraphics[width=0.28\textwidth]{figures//basketball-another-subspaces.eps}
%         \label{fig:basketball-projection1}
%     }
% %    \caption{Histogram representation of a group on three single features}
%     \label{fig:Basketball-Example}
% \end{figure}

% %%==========================================================================================
% % \begin{note}
% % We use $\rho_s(\cdot)$ to describe an outlying scoring function.
% % $\rho_s(\cdot)$ quantifies the outlying degree of the query group in a subspaces $s$.
% % In addition,
% % we use the scoring function to make a descending order and at last
% % filter out the top k group outlying subspaces.
% % It is obvious that the outlying subspaces make the
% % query group different from other groups.
% % \end{note}
% %%==========================================================================================

% \end{slide}
% %%
% %%==========================================================================================


% %%==========================================================================================
% %%
% \begin{slide}[toc=,bm=]{Term Definition}
% \begin{itemize}
% \item
% Trivial Outlying Features

% \begin{itemize}
% \item
% \smallskip
% One-dimension subspaces.

% \item
% ${G_q}$'s outlying degree $\rho(\cdot)$ $>$ $\alpha$.
% \end{itemize}
% \end{itemize}
% \begin{table}
% \setlength{\abovecaptionskip}{0pt}
% \setlength{\belowcaptionskip}{10pt}
% \centering
% \caption{$\alpha = 4$}

% \begin{tabular}{  c  |  c }
% \toprule
% \centering
% \texttt{Feature}  & \texttt{Outlying Degree}  \\
% \midrule
%  {\textcolor{orange}{\{$F_1$\}}} & $4.351$ \\
%  {\{$F_3, F_4$\}}                & $4.024$ \\
%  {\{$F_2, F_4$\}}                & $2.318$ \\
%  {\{$F_2$\}}                     & $2.002$ \\
%  {\{$F_3$\}}                     & $1.028$ \\
% \bottomrule
% \end{tabular}
% \end{table}

% %%==========================================================================================
% \begin{note}
% In order to identify the top-k outlying subspaces,
% we categorize the features into $2$ non-overlapping groups,
% trivial outlying features and non-trivial outlying subspaces.

% First, let me introduce the trivial outlying features.

% Trivial outlying features are one-dimension subspaces.
% In the subspace,
% the query group's outlying degree is larger than the threshold $\alpha$.

% We can see from table $1$,
% when the specified threshold $\alpha = 4$,
% the trivial outlying feature is \{$F_1$\}.
% \end{note}
% %%==========================================================================================

% \end{slide}
% %%
% %%==========================================================================================


% %%==========================================================================================
% %%
% \begin{slide}[toc=,bm=]{Term Definition}
% \begin{itemize}
% \item
% Non-Trivial Outlying Subspaces
% \begin{itemize}
% \item
% \smallskip
% Multi-dimension subspaces.

% \item
% \smallskip
% ${G_q}$'s outlying degree $\rho(\cdot)$ $>$ $\alpha$.
% \end{itemize}
% \end{itemize}

% \begin{table}
% \setlength{\abovecaptionskip}{0pt}
% \setlength{\belowcaptionskip}{10pt}
% \centering
% \caption{$\alpha = 4$}

% \begin{tabular}{  c  |  c }
% \toprule
% \centering
% \texttt{Feature}  & \texttt{Outlying Degree}  \\
% \midrule
%  {\{$F_1$\}}                           & $4.351$ \\
%  {\textcolor{orange}{\{$F_3, F_4$\}}}  & $4.024$ \\
%  {\{$F_2, F_4$\}}                      & $2.318$ \\
%  {\{$F_2$\}}                           & $2.002$ \\
%  {\{$F_3$\}}                           & $1.028$ \\
% \bottomrule
% \end{tabular}
% \end{table}

% %%==========================================================================================
% \begin{note}
% Next,
% I will introduce the non-trivial outlying subspaces.
% Non-Trivial outlying subspaces are multi-dimension subspaces.
% In the subspace,
% the query group's outlying degree is larger than the threshold $\alpha$.

% Table $2$ shows that,
% when the threshold $\alpha$ equal four,
% the non-trivial outlying subspace is \{$F_3$, $F_4$\}.
% \end{note}
% %%==========================================================================================

% \end{slide}
% %%
%%==========================================================================================


\section{Planning}


%%==========================================================================================
%%
\begin{slide}{Planning}

%   \begin{table}
%     % \setlength{\abovecaptionskip}{0pt}
%     % \setlength{\belowcaptionskip}{10pt}
%     \centering
%     \caption{Data}  


%   \begin{tabular}{c c c c}
%     \hline
%      & id & cuisine	& ingredients    \\ 
%     \hline
%     0 & 10259 &	greek	& 'romaine lettuce', 'black olives', 'grape tom...\\
%     1	& 25693	& southern_us	& 'plain flour', 'ground pepper', 'salt', 'toma...\\
%     2	& 20130	& filipino & 'eggs', 'pepper', 'salt', 'mayonaise', 'cooki...\\
%     3	& 22213	& indian & 'water', 'vegetable oil', 'wheat', 'salt'\\
%     4	& 13162	& indian & 'black pepper', 'shallots', 'cornflour', 'cay...\\           
%     \hline

%   \end{tabular}
% \end{table}  

\begin{itemize}
  \item About the study plan for next week:\\
  \begin{itemize}
    \item Watch 5 Flip02 learning videos.\\
    \item Read an anomaly detection related literature.\\
    \item Continue to watch the video course Data Mining: Theory and Algorithm Course.\\
    \item Collect information about DNS encrypted traffic identification.\\
  \end{itemize}          
\end{itemize}



\end{slide}
%%
%%==========================================================================================
% \begin{slide}[toc=,bm=]{data analysis}
%   \begin{itemize}
%     \item  The percentage of dishes from each country in the total training set:\\
%   \end{itemize}
%   \begin{table} 
%     %\begin{center}
%       \includegraphics[width=1.0\linewidth,height=.4\linewidth]{C:/Users/lenovo/Desktop/01/slides/Figure/Figure_1.eps}\
%         {image1:The percentage of dishes}\
%   \end{table}
%   %%==========================================================================================
%   % \begin{note}
%   % The second challenge is how to evaluate the outlying degree of
%   % the query group between different aspects.
%   % In that case,
%   % we need to design a scoring function to measure the outlying degree.
%   % But adopting an appropriate scoring function without dimension bias still remains a problem.
%   % \end{note}
%   %%==========================================================================================
  
%   \end{slide}

%   \begin{slide}[toc=,bm=]{}
%     \begin{itemize}
%       \item  The following results are obtained by sorting and sorting the data of ingredients:\\
%     \end{itemize}

%     \begin{table}
%       % \setlength{\abovecaptionskip}{0pt}
%       % \setlength{\belowcaptionskip}{10pt}
%       \centering
%       \caption{Data}  
  
  
%     \begin{tabular}{c c | c c | c c }
%       \hline
%       id & cuisine & id & cuisine & id & cuisine \\ 
%       \hline
%       salt & 18049 & sugar & 6434 & pepper & 4438\\
%       onions	& 797 & garlic cloves & 6237 & vegetable oil & 4385\\
%       olive oil	& 7972 & butter & 4848 & eggs & 3388\\
%       water	& 7457 & ground black pepper & 4785 & soy sauce & 3296\\
%       garlic	& 7380 & all-purpose flour & 4632 & kosher salt & 3113\\           
%       \hline
  
%     \end{tabular}
%   \end{table}  

% \begin{itemize}
%   \item  We can also get the dish label from the training set,as follows: \\
%   'greek', 'southern_us', 'filipino', 'indian', 'jamaican', 'spanish', 'italian', 'mexican', 'chinese', 'british', 'thai', 'vietnamese', 'cajun_creole', 'brazilian', 'french', 'japanese', 'irish', 'korean', 'moroccan', 'russian'\\
% \end{itemize}
    
%     \end{slide}




% \begin{slide}{Data Processing}

  

% \begin{table}
%   % \setlength{\abovecaptionskip}{0pt}
%   % \setlength{\belowcaptionskip}{10pt}
%   \centering
%   \caption{Data}  


% \begin{tabular}{c c c c}
%   \hline
%    & Column & Non-Null Count & Dtype    \\ 
%   \hline
%   0 & id & 39774 non-null & int64 \\
%   1 & cuisine & 39774 non-null & object  \\
%   2 & ingredients & 39774 non-null & object  \\          
%   \hline

% \end{tabular}
% \end{table}  

% \begin{itemize}
% \item
% Existing Methods - \textcolor{orange} {Score-and-search}

% \begin{itemize}
% \item
% Define an outlying score function.

% \item
% Search subspaces.
% \end{itemize}
% \bigskip
% \twocolumn[
% \savevalue{lfrheight}=5cm,
% \savevalue{lfrprop}={
% linestyle=solid,framearc=.2,linewidth=1pt},
% rfrheight=\usevalue{lfrheight},
% rfrprop=\usevalue{lfrprop}
% ]{
% Disadvantages
% \begin{itemize}
% \item
% \smallskip
% Dimensionality bias.

% \item
% \smallskip
% Search efficiency is \textcolor{orange}{Not} high (dataset is large).

% \item
% \smallskip
% \textcolor{orange}{Not} identify group outlying aspects.
% \end{itemize}
% }{
% Advantages
% \begin{itemize}
% \item
% \smallskip
% Quantify the outlying degree correctly.

% \item
% \smallskip
% High Comprehensibility.

% \end{itemize}
% }
% \end{itemize}
% \begin{itemize}
%   \item 
% \end{itemize}

% %%==========================================================================================
% % \begin{note}
% % For score-and-search method,
% % it defines an outlying score function,
% % and then searches for each subspace until it finds out a subspace that
% % makes the query point show the best score.

% % The advantages of this method are:
% % first, it enables to quantify the outlying degree accurately.
% % Besides, its comprehensibility is high.

% % While the disadvantages include three main aspects:
% % the first one is it has dimensionality bias;
% % Additionally, its search efficiency is low.
% % Last but not least, it doesn't identify group outlying aspects.
% % \end{note}
% %%==========================================================================================

% \end{slide}
%%
%%==========================================================================================


%%==========================================================================================
%%
% \begin{slide}{Data Preprocessing}
% \begin{table} 
%     %\begin{center}
%       \includegraphics[width=1.0\linewidth,height=.4\linewidth]{C:/Users/lenovo/Desktop/01/code/figure/Figure_1.eps}\
%         {image1:Ingredients in a Dish Distribution}\
% \end{table}

% \begin{itemize}
%   \item The maximum ingredient in the dishes in the training set is 65.\\
%         The minimum and medium ingredients in the training set are 1. 
% \end{itemize}
% % \twocolumn
% % {
% % Group Outlying Aspects Mining
% % \begin{itemize}
% % \item
% % \smallskip
% % Focus on differences between \textcolor{orange}{groups}.

% % \item
% % \smallskip
% % \textcolor{orange}{Multiple} points.
% % \medskip
% % \end{itemize}
% % \vspace{0.75cm}
% % %\vspace{0.1cm}
% % \begin{figure}
% %   \centering
% %   \selectcolormodel{rgb}
% %   \missingfigure{Testing a long text string.}
% %   %\includegraphics[width=0.6\textwidth]{figures//example-basketball-projection.eps}\\
% %   \caption{Group Outlying Aspects Target}\label{fig:GroupOutAspect-target}
% % \end{figure}
% % }
% % {
% % Outlying Aspects Mining
% % \begin{itemize}
% % \item
% % Concentrates on differences between \textcolor{orange}{objects}.

% % \item
% % \textcolor{orange}{One} point.
% % \end{itemize}
% % \bigskip
% % \begin{figure}
% %   \centering
% %   \selectcolormodel{rgb}
% %   \missingfigure{Testing a long text string.}
% % %  \includegraphics[width=0.5\textwidth]{figures//OutAspect_target.eps}\\
% %   \caption{Outlying Aspects Target}\label{fig:OutAspect-target}
% % \end{figure}
% % }

% %%==========================================================================================
% % \begin{note}
% % In this research paper,
% % we proposed the group outlying aspects mining.
% % Now,
% % let me summarize the differences between group outlying aspects mining and outlying aspects mining.

% % Group outlying aspects mining mainly focuses on the differences between groups.
% % But outlying aspects mining mainly concentrates on the differences between objects.
% % The target of group outlying aspects mining can be seen as many points.
% % While the target of outlying aspects mining can be regarded as one point.

% % In the NBA example,
% % group outlying aspects mining focuses on the advantages
% % or disadvantages of one team,
% % however,
% % outlying aspects mining focuses on the advantages or disadvantages of one player.
% % \end{note}
% %%==========================================================================================

% \end{slide}
% %%
% %%==========================================================================================


% %%==========================================================================================
% %%
% % \begin{slide}[toc=,bm=]{}
% %Challenges (1)
% % \begin{itemize}
% % \item
% % How to \textcolor{orange}{represent} the group features.

% % \begin{itemize}
% % \item
% % Can be affected by outlier values.

% % \item
% % Can \textcolor{orange}{Not} reflect the overall distribution of group features.
% % \end{itemize}
% % \end{itemize}

% %%==========================================================================================
% % \begin{note}
% % Based on current existing methods,
% % there still remains some research challenges.

% % The first one is how to represent the group features
% % based on the features of the individuals in the group.

% % Although the arithmetic mean of all elements
% % in each feature can describe the features of one group.
% % It can be affected by outlier values,
% % and can't reflect the entire distribution of group features.L
% % \end{note}
% %%==========================================================================================

% % \end{slide}
% %%
% %%==========================================================================================


% %%==========================================================================================
% %%
% \begin{slide}[toc=,bm=]{}

% % \begin{itemize}
% % \item
% % How to \textcolor{orange}{evaluate} the outlying degree in different aspects.


% % \end{itemize}
% \begin{table} 
%   %\begin{center}
%     \includegraphics[width=1.0\linewidth,height=.4\linewidth]{C:/Users/lenovo/Desktop/01/code/figure/Figure_2.eps}\
%       {image2:The kind of dishes}\
% \end{table}
% %%==========================================================================================
% % \begin{note}
% % The second challenge is how to evaluate the outlying degree of
% % the query group between different aspects.
% \begin{itemize}
% \item There are 20 different categories (dishes), which may be a multi-category classification problem.
% \end{itemize}
% % In that case,
% % we need to design a scoring function to measure the outlying degree.
% % But adopting an appropriate scoring function without dimension bias still remains a problem.
% % \end{note}
% %%==========================================================================================

% \end{slide}
% %%
% %%==========================================================================================


% %%==========================================================================================
% %%
% \begin{slide}[toc=,bm=]{}

% \begin{itemize}
% \item
% How to \textcolor{orange}{improve} the efficiency.

% \begin{itemize}
%   \begin{table} 
%     %\begin{center}
%       \includegraphics[width=1.0\linewidth,height=.4\linewidth]{C:/Users/lenovo/Desktop/01/code/figure/Figure_3.eps}\
%         {image3:Top 20 ingredients with the most usage}\\
%   \end{table}
% % \item
% When the dimension of the \textcolor{orange}{data is high},
% the candidate subspace grows exponentially.
% \begin{itemize}
%   \item Salt is used in most dishes, and these may not be the main features for identifying dishes.
% \end{itemize}
% \item
% It will easily go beyond the limits of the computation resources.

% \end{itemize}
% \end{itemize}

%%==========================================================================================
% \begin{note}
% The third challenge is how to improve efficiency.

% To be specific,
% when the dimension of data is high,
% the candidate subspace increase dramatically,
% so that it is very easy to exceed the limit of computer resources.
% \end{note}
%%==========================================================================================

% \end{slide}
% %%
% %%==========================================================================================


% \section{Data Preprocessing}


% % %%==========================================================================================
% % %%
% \begin{slide}[toc=,bm=]{Data Preprocessing}
%   First process the string:
% % Framework of GOAM algorithm:
% \begin{itemize}
%   \item To remove everything except a-z and A-Z and to make list element a string element. 
% \end{itemize}
% % \bigskip
% \begin{table}
%   \centering
%   \caption{Data}  

% \begin{tabular}{c c c c c }
%   \hline
%   id & cuisine & ingredients \\
%   \hline
%    0 & greek & romaine lettuce black olive grape tomato garli... \\
%    1 & southern_us & plain flour pepper salt tomato black pepper ... \\
%    2 & filipino & egg pepper salt mayonaise cooking oil green ch... \\
%    3 & indian & water vegetable oil wheat salt \\
%    4 & indian & black pepper shallot cornflour cayenne pepper ... \\
%  \hline
% \end{tabular}
% \end{table} 
% % \begin{itemize}
% %   \item Use cosine similarity to judge the distance between two dishes.
% % \end{itemize}
% \end{slide}
% %%
% %%==========================================================================================


% %%==========================================================================================
% %%
% \begin{slide}{TFiDF Vectorizer}
%   TFiDF Vectorizer on the processed data:
%   \begin{itemize}
%     \item Use TFiDF Vectorizer to evaluate the importance of each dish of vegetable raw materials \\
%   \end{itemize}
%   \begin{table} 
%     %\begin{center}
%       \includegraphics[width=0.8\linewidth,height=.4\linewidth]{C:/Users/lenovo/Desktop/01/slides/Figure/Figure_4.eps}\\
%         {image2:TFiDF}\
%   \end{table}

% \end{slide}
% %%
% %%==========================================================================================


% %%==========================================================================================
% %%
% \begin{slide}{similar dishes}
%   More similar dishes and visualize dishes:\\
%   From this picture, We can notice there are 3 clusters of cuisines.
%   \begin{table} 
%     %\begin{center}
%       \includegraphics[width=1.0\linewidth,height=.4\linewidth]{C:/Users/lenovo/Desktop/01/slides/Figure/Figure_2.eps}\
%         {image3:More similar dishes}\
%   \end{table}  

% \end{slide}
% %%
% %%==========================================================================================
\section{Thanks for watching}


%==========================================================================================
%
% \begin{slide}[toc=,bm=]{Model}
%  Use Linear SVC model for prediction:\\
%  \begin{itemize}
%   \item Here we use the value obtained by the tfidf method to predict. \\
%   \item best_score:0.7857898061559815 
% \end{itemize}
% \end{slide}
% % %%==========================================================================================
% % %%
% \begin{slide}[toc=,bm=]{Model}
%   % Use Linear SVC model for prediction:\\
%   \begin{itemize}
%    \item C=604.5300203551828
%          gamma=0.9656489284085462 \\
%   \item Predict the variables, the prediction results are as follows:\\
%   \begin{table} 
%     %\begin{center}
%       \includegraphics[width=1.0\linewidth,height=.4\linewidth]{C:/Users/lenovo/Desktop/01/slides/Figure/Figure_5.eps}\
%         {image4:forecast result}\
%   \end{table} 
%   % \item 
%  \end{itemize}
%  \end{slide}

\end{document}

\endinput
